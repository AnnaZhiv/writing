\documentclass{article}
\usepackage[T1]{fontenc}
\usepackage{graphicx}
\usepackage[english, russian]{babel}
\begin{document}

самая простая вставка без опций \\
\includegraphics{mem} 
\newpage

%%%%%%%%%%%%%%%%%% ширина и высота абсолютные %%%%%%%%%%%%%%%%%%%%%%%

задали высоту в абсолютной величине 2 см \\
\includegraphics[height=2cm]{mem} 
\\
задали ширину в абсолютной величине 2 см \\
\includegraphics[width=2cm]{mem} %
\\
задали высоту и ширину в абсолютной величине 2 см \\
\includegraphics[height=2cm, width=2cm]{mem} 
\\
%%%%%%%%%%%%%%%%%% ширина и высота относительные (одна колонка) %%%%%%%%%%%%%%%%%%%%%%%
задали высоту в относительной величине 15\% от высоты текста \\
\includegraphics[height=0.15\textheight]{mem} 
\\
задали ширину относительной величине 15\% от ширины текста \\
\includegraphics[width=0.15\textwidth]{mem}  
\\
задали высоту и ширину в относительной величине 15\% от текста \\
\includegraphics[height=0.15\textheight, width=0.15\textwidth]{mem} 
\\
задали ширину относительной величине 15\% от ширины линии \\
\includegraphics[width=0.15\linewidth]{mem} 
\newpage
%%%%%%%%%%%%%%%%%% окружения %%%%%%%%%%%%%%%%%%%%%%%
в окружении center \\
\begin{center}
\includegraphics[height=2cm, width=2cm]{mem} 
\end{center}

в окружении flushleft \\
\begin{flushleft}
\includegraphics[height=2cm, width=2cm]{mem} 
\end{flushleft}

в окружении flushright \\
\begin{flushright}
\includegraphics[height=2cm, width=2cm]{mem} 
\end{flushright}
\newpage
%%%%%%%%%%%%%%%%%% clip trip rotate scale %%%%%%%%%%%%%%%%%%%%%%%

\includegraphics[width=0.2\linewidth,  trim = 0 0 50 50]{mem} % trim =dl db dr du

\includegraphics[width=0.2\linewidth,  clip]{mem} % clip 

\includegraphics[scale=0.2]{mem} % scale 

\includegraphics[width=0.2\linewidth,  angle=60]{mem} % rotate 

\end{document}